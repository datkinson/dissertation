%\addcontentsline{toc}{chapter}{Development Process}
\chapter{Development Process}

You need to describe briefly the life cycle model that you used. Do not force your project into the waterfall model if it is better described by prototyping or some other evolutionary model. You do not need to write about all of the different process models that you are aware of. Focus on the process model that you have used. It is possible that you needed to adapt an existing process model to suit your project; clearly identify what you used and how you adapted it for your needs.

In most cases, the agreed objectives or requirements will be the result of a compromise between what would ideally have been produced and what was felt to be possible in the time available. A discussion of the process of arriving at the final list is usually appropriate.

You should briefly describe the design method you used and any support tools that you used. You should discuss your choice of implementation tools - programming language, compilers, database management system, program development environment, etc.

\section{Introduction}
I chose to use the iterative and incremental approach to development.  This is mainly because of how modular my project is.  In theory, I can add more functionality with minor adjustments to the core system, thus making iterative/incremental very suited to my needs.
\\Each part of the system in an incremental strategy can be developed independently and slotted together as they reach completion.
\\Each iteration is a review of the previous which has been reworked and improved upon.
\\For a well functioning system it needs good design, quality programming and a good debugging process.  So, after designing the inital system, writing a simple prototype it is then time for debugging it to get an indication of what the main flaws are.  Once these flaws have been clearly identified a new design has to be drawn up to correct these issues.  After writing the new version following the revised design the cycle continues in the same manner, design, write then debug.


\section{Modifications}
No real modifications were made to this development process as it works for individuals and for teams without alteration.

\section{Version Control}
This is majorly usefull in any project that involves managing code or documents digitaly.  It is even usefull as a backup tool, to be safe from accidental deletions, hard drive failure or any number of unfortunate occurances as you can just re-download the files.
\\The other features that version control systems offer as of more use in this type of project.  Branching and merging are two of the most used features.  These enable the user to make a branch within the project in which they can work on a specific feature independantly of the main project.  You may make multiple branches at the same time and work on different things all independant of each other.  Once these are finished you can merge them back into the main project, this is a very nice feature version control systems offer as it performs most, if not all of this for you, instead of having to manualy try and integrate each line of the branch files back into the main ones.
\\I have chosen to use Git for developing this project due to how powerful the merge feature is as well as a website called github \cite{github} which will host repositories for people.  The website also has nice usage statistics and offer some private repositories to students.  Github repositories are normally open to the general public unless you pay a fee for having non public facing ones, being a student enables me to have a small number of these private ones which let em control when I am ready to release a project to public viewing.

