\chapter{Testing}

%Detailed descriptions of every test case are definitely not what is required here. What is important is to show that you adopted a sensible strategy that was, in principle, capable of testing the system adequately even if you did not have the time to test the system fully.

%Have you tested your system on 'real users'? For example, if your system is supposed to solve a problem for a business, then it would be appropriate to present your approach to involve the users in the testing process and to record the results that you obtained. Depending on the level of detail, it is likely that you would put any detailed results in an appendix.

\section{Overall Approach to Testing}
To test all aspects of a system is always a good thing to do, idealy all very thouroughly.  Time does not always permit this but at least some general testing of all systems should be caried out to highlight any obvious problems.
\section{Automated Testing}
Due to the embedded nature of this project, automated testing is very limited.  As there is no simulator for an arduino microcontroller and all the various components that can be used in conjunction with one, no automated testing of this can be achieved without a far more complex hardware system which somehow can test other hardware configurations.  The only part of the system that can be automatically tested with any degree of reliability is to parse the software code and check is it is syntactically valid.
\subsection{Unit Tests}
These are written to test each individual part fo the project.  This starts with testing if the code validates.  A compiler is used to test this and a compiler is another program that checks if code will actually work, not neccesarily for its intended purpose but just check that it does not have a major flaw that will stop the finished program from running at all and converts it into either a binary capable of being run as an executable program or into another language.  This compiled code is what will run on the hardware itself rather than through an interpreter which is another program that interprets what the code is trying to do and runs that.
\\The next stage if unit tests is to check that the software for each individual hardware interaction works as expected.  This includes checking that when the code tells a motor to turn in a specific direction that it actually turns as it is told to.  A motor is laid out on a workbench, hooked up to the microcontroller and told to turn.  If it does as expected then the test will be considered as passed, otherwise it will be considered as failed and debugging will be required before running the tests again.
\subsection{User Interface Testing}

\subsection{Stress Testing}

\section{Integration Testing}

\section{User Testing}
