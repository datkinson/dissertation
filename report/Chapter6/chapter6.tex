\chapter{Evaluation}

%Examiners expect to find in your dissertation a section addressing such questions as:

%\begin{itemize}
%   \item Were the requirements correctly identified?
%   \item Were the design decisions correct?
%   \item Could a more suitable set of tools have been chosen?
%   \item How well did the software meet the needs of those who were expecting to use it?
%   \item How well were any other project aims achieved?
%   \item If you were starting again, what would you do differently?
%\end{itemize}

%Such material is regarded as an important part of the dissertation; it should demonstrate that you are capable not only of carrying out a piece of work but also of thinking critically about how you did it and how you might have done it better. This is seen as an important part of an honours degree.
%There will be good things and room for improvement with any project. As you write this section, identify and discuss the parts of the work that went well and also consider ways in which the work could be improved.
%The critical evaluation can sometimes be the weakest aspect of most project dissertations. We will discuss this in a future lecture and there are some additional points raised on the project website.

The final product met what was outlined in the abstract.  This being that a robot was built that could move around an environment under it's own power, detect and avoid approaching obstacles.
\\\\It does not perform this task as reliably as it was expected to from the initial description, the main issue being with the drive system.  I had under estimated how much torque the motors would require to move the total weight of the finished robot, this was not identified correctly in the initial design.  This issue was addressed later on in the project implementation but was not fully corrected, the robot does move but it only takes a moderate incline like a ramp to make the robot struggle to move again.  Choosing stepper motors alone was a bad design decision and in hindsight I should have used a gear system to gain the extra torque that it needed to handle movement up an incline.
\\\\The sensor pair worked well to validate each other in the event of interference or an unfavourable environment.  With the robot only being able to see in two very narrow straight lines out in front of it, I would have liked to add more sensor pairs to get a wider view of the environment.  A wider view would allow for better calculating which direction had the least obstacles in and to move accordingly.
\\Extra sensor pairs would be arranged in a circle around the robot to form a halo.  This would still only give narrow points of view of the environment.  I could possibly mount a sensor pair on a servo controller pan and tilt mount, this could then point the sensors at any angle from the robot and take readings.  This would work but it would also be very slow and impractical.  A camera would also be a good solution I would consider for generating a more accurate map of the robot's surrounding environment to better work out where to go.  The camera solution will require far more processing than the current sensor solution but has a much higher resolution.
\\\\I think that the tools I used to complete this project were appropriate, these being the Arduino development environment, version control system used to develop the software and the various hardware tools used to create the physical robot.  A different version control system could have been used to maintain the code such as subversion or mercurial but this project did not actually require any complex features of these systems apart from merging.  All of the merges that took place were simple and very easy to do, but that may have been due to how good Git is at performing this function.
\\As stated previously the project aims were met but not to as high of a standard as I had hoped for.  The robot has to stop moving to take sensor readings as the microcontroller is not capable of multitasking, this is an issue as it puts more strain on the motors with this stop and start behaviour.
\\\\If I were to start this project again from scratch I would seperate the main tasks such as motor control, sensor readings and decision making each to their own processor.  By this I mean that each subsystem would have it's own microcontroller and a central unit would tell them what to do.  This would mean that the control unit can tell the robot to move forwards unti told otherwise and the sensor subsystem could take all the readings it wants without disrupting the movements of the robot.  I would also use a sensor system with higher resolution of the environment and which is more reliable such as a camera or a pair of cameras to be able to detect depth.  A laser scanner would also be a good sensor to have but they are very expensive and require even more processing than the camera.
\\The robot is very heavy due to the amount of aluminium used in the construction of the chassis and the mounts, this could be greatly reduced by using the 3D printer to fabricate a chassis.  This chassis would be different from one made out of cut sheets of plastic because it can be made hollow with an internal structure such as a honeycomb weave to keep the strength near to that of a solid sheet.  This method would reduce the cost and be made to whichever specifications are required.
\\\\Overall the project met it's basic aim but has much room for improvment and expansion by building upon all of the mentioned system alterations.