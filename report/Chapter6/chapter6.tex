\chapter{Evaluation}

%Examiners expect to find in your dissertation a section addressing such questions as:

%\begin{itemize}
%   \item Were the requirements correctly identified?
%   \item Were the design decisions correct?
%   \item Could a more suitable set of tools have been chosen?
%   \item How well did the software meet the needs of those who were expecting to use it?
%   \item How well were any other project aims achieved?
%   \item If you were starting again, what would you do differently?
%\end{itemize}

%Such material is regarded as an important part of the dissertation; it should demonstrate that you are capable not only of carrying out a piece of work but also of thinking critically about how you did it and how you might have done it better. This is seen as an important part of an honours degree.
%There will be good things and room for improvement with any project. As you write this section, identify and discuss the parts of the work that went well and also consider ways in which the work could be improved.
%The critical evaluation can sometimes be the weakest aspect of most project dissertations. We will discuss this in a future lecture and there are some additional points raised on the project website.

The final product met what was outlined in the abstract.  This being that a robot was built that could move around an environment under it's own power, detect and avoid approaching obstacles.
\\\\It does not perform this task as reliably as it was expected to from the initial description, the main issue being with the drive system.  I had under estimated how much torque the motors would require to move the total weight of the finished robot, this was not identified correctly in the initial design.  This issue was addressed later on in the project implementation but was not fully corrected, the robot does move but it only takes a moderate incline like a ramp to make the robot struggle to move again.  Choosing stepper motors alone was a bad design decision and in hindsight I should have used a gear system to gain the extra torque that it needed to handle movement up an incline.
\\\\The sensor pair worked well to validate each other in the event of interference or an unfavourable environment.  With the robot only being able to see in two very narrow straight lines out in front of it, I would have liked to add more sensor pairs to get a wider view of the environment.  A wider view would allow for better calculating which direction had the least obstacles in and to move accordingly.
\\Extra sensor pairs would be arranged in a circle around the robot to form a halo.  This would still only give narrow points of view of the environment.  I could possibly mount a sensor pair on a servo controller pan and tilt mount, this could then point the sensors at any angle from the robot and take readings.  This would work but it would also be very slow and impractical.
