\chapter{Background \& Objectives}

I was first exposed to electronics in an academic enviroment in high school.  This was only very basic circuitry such as making a light flash to using simple integrated circuits.  Being introduced to integrated circuits made building an electonic timer much easier, which was the first thing I produced using these small chips.
\\Fast forward to college five years later and I am still interested in electronics.  Still using these wonderous little chips to build more interesting circuits I built an audio amplifier whereby I input a waveform into the circuit, either generated by a signal generator or my guitar, and amplify it or smooth the signal to create a new sound, then output this amplified signal to a speaker.
\\At college I also took a computing class in which the programming language Visial Basic was taught as part of the course.  Naturally the next step would be to compine the electronics with the programming knowledge.  This took the form of a small blinking light project where I use a PIC (Peripheral Interface Controller) to flash an LED (Light Emitting Diode).  A PIC is a small chip (Integrated Circuit) that can run small ammounts of code to read inputs and control outputs on its various pins.
\\In the summer between the end of College and starting University I discovered a range of open source hardware microcontrollers called Arduino.  These boards made combining program code and electonic hardware much easier by doing much of the base work for me.  These microcontrollers have a large community, having written all forms of libraries to interface the board with various pieces of hardware and control them with much less effort than would be needed when using a PIC.


\textbf{Note: All of the sections and text in this example are for illustration purposes. The main Chapters are a good starting point, but the content and actual sections that you include are likely to be different.}
