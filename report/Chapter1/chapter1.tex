\chapter{Background \& Objectives}

I was first exposed to electronics in an academic environment in high school.  This was only very basic circuitry, such as making a light flash by using simple integrated circuits.  Being introduced to integrated circuits made building an electronic timer much easier, which was the first thing I produced using these small chips.  This was very satisfying when it finally worked, a feeling I still get when something I make that works as intended.
\\Fast forward to college five years later and I am still fascinated by electronics.  Still using these wondrous little chips to build more interesting circuits I built an audio amplifier whereby I input a waveform into the circuit, either generated by a signal generator or my guitar, and amplify it or smooth the signal to create a new sound, then output this amplified signal to a speaker.  This distorted sound is similar to those created by a guitar amplifier that has built in effects or an specific effects pedal also used by guitarists.
\\At college I also took a computing class in which the programming language Visual Basic was taught as part of the course.  Naturally the next step would be to combine the electronics with the programming knowledge.  This took the form of a small blinking light project where I use a PIC (Peripheral Interface Controller) to flash an LED (Light Emitting Diode).  A PIC is a small chip (Integrated Circuit) which can run small amounts of code to read inputs and control outputs on its various pins.
\\In the summer between the end of College and starting University I discovered a range of open source hardware microcontrollers called Arduino.  These boards made combining program code and electronic hardware much easier by doing much of the base work for me.  These microcontrollers have a large community, which has written all forms of libraries to interface the board with various pieces of hardware and control them with much less effort than would be needed when using a PIC.  The PIC does have a large variety of libraries but the Arduino seems to have a much wider variety of what is supported and a very active community to help if you get stuck.
\\I have also had some experience using the pioneer research robot created by Adept MobileRobots \cite{mobilerobots} which are used by Aberystwyth University in the robotics lab.  The experience with these robots was to use their ultrasonic sensors to try and avoid hitting some polystyrene boards.  Due to the limited time available to use these robots the resulting code was not very effective or polished, but it has heavily influenced my ideas for designing my current project and further pricked my enthusiasm for robotics and all of the possible applications it has.
\\\\My main objective with this project is to produce a piece of hardware that can manoeuvre itself around an environment under it's own power without bumping into anything.  This is to be built utilising the knowledge I have gained about electronics and programming from previous projects and from the courses I have attended as part of my University degree.
